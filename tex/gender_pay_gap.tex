\documentclass[12pt]{article}
\usepackage[utf8]{inputenc}
\usepackage[margin=1in]{geometry}
\usepackage{amssymb}
\usepackage{amsmath}
\usepackage{amsthm}
\usepackage{bbm}
\usepackage{mathtools}
\usepackage{dsfont}
\usepackage{color}
\usepackage{enumerate}
\usepackage[toc,page]{appendix}
\usepackage{bm}
\usepackage{setspace}
\usepackage{multirow}
\usepackage{hyperref}
\usepackage{array}
\usepackage{xcolor}
\usepackage{eurosym}
\usepackage{longtable}

\newcolumntype{C}[1]{>{\centering\arraybackslash}p{#1}}

\hypersetup{
  colorlinks   = true, %Colours links instead of ugly boxes
  urlcolor     = blue, %Colour for external hyperlinks
  linkcolor    = blue, %Colour of internal links
  citecolor   = blue %Colour of citations
}

\usepackage{float}
\usepackage[format=plain,
            labelfont={bf, up},
            textfont=up]{caption}
\usepackage{subcaption}
\usepackage{graphicx}

\setlength\parindent{3ex}
\allowdisplaybreaks
\onehalfspacing

\newcommand*{\tabindent}{ \hspace{3mm}}

\usepackage{natbib}
\bibliographystyle{plainnat}
\setcitestyle{authoryear,open={(},close={)}}


\definecolor{Purple}{HTML}{414487}
\definecolor{Teal}{HTML}{2A788E}
\definecolor{Green}{HTML}{7AD151}
\definecolor{Yellow}{HTML}{FDE725}
\definecolor{Orange}{HTML}{DE7065}

\DeclareMathOperator*{\argmax}{arg\,max}

\title{The Gender Wage Gap: Literature}
\author{Pauline Corblet}
\date{\today}

\begin{document}
  \maketitle

  \section{The Gender Wage Gap}
  \cite{blauGenderWageGap2017}: overall survey of explanations using the PSID + survey of recent literature. Long-term trend: substantial reduction in gender wage gap. But the extent differs depending on what part of the wage distribution we look at. Average female to male ratio in 2014: 79\%. What plays a role:
  \begin{itemize}
    \item Gender differences in \textbf{occupation and industry}
    \item Differences in \textbf{gender roles and the division of labor}.
    \item \textbf{Compensating differential} (women accept lower wage in exchange for other benefits).
    \item \textbf{Discrimination}.
  \end{itemize}
  What is not very important (anymore):
  \begin{itemize}
   \item Human capital variables (education and work experience).
   \item Differences in non-cognitive skills.
  \end{itemize}
  Specific mechanisms:
  \begin{itemize}
  \item \cite{bertrandDynamicsGenderGap2010}: gender wage gap among young MBA graduates due to difference in training, \textbf{hours worked} and women \textbf{career interuptions}. 
    \item \cite{goldinGrandGenderConvergence2014}, \cite{wiswallPreferenceWorkplaceInvestment2018}: \textbf{Time flexibility} drives the gender wage gap. Also \cite{denningReturnHoursWorked2022}, \cite{wassermanHoursConstraintsOccupational2022} \cite{mannasooWorkingHoursGender2022}: hours worked.
    \item \cite{blauGenderInequalityWages2012} in chapter 1: \textbf{Between firm}: women tend to match with low paying firms, men with high paying firms. Explains it through discrimination. Also \cite{webberFirmLevelMonopsonyGender2016} (monopsony), \cite{cardBargainingSortingGender2015} (sorting and bargaining), \cite{theodoropoulosAreWomenDoing2022} (share of female managers reduces gender pay gap).
    \item \cite{dasWomenPipelineDynamic2023}: \textbf{Within firm}: pipeline effect: women have to catch up on men in high ranking, high paying positions.
    \item \cite{fortinGenderRoleAttitudes2015}: Stalled progression of \textbf{gender role attitudes} impacted female labor force participation. Also \cite{kamalAttitudesPersonalityAustralian2022}, \cite{siminskiSpecializationComparativeAdvantage2022}: test a Beckerian model, by which discrimination induced gender division of labor
    \item \cite{reubenTasteCompetitionGender2019}: \textbf{Taste for competition} accounts for 10\% of the gender wage gap among MBA graduates.
    \item \cite{adamecz-volgyiGenderGapTop2022}, \cite{lavettiGenderDifferencesSorting2022}, \cite{kamalAttitudesPersonalityAustralian2022}, \cite{exleyGenderGapSelfPromotion2022}: \textbf{Psychological traits}, like over-confidence or risk attitudes, helps men getting higher wages.
    \item \cite{barbanchonGenderDifferencesJob2021}: Women value \textbf{commuting time} about 20\% more than men. Accounts for 14\% of the residualized wage gap.
    \item \cite{burbanoGenderGapMeaningful2022}: Women sort more into \textbf{‘meaningful' work} (that have high pro-social impact). Accounts for 1/3 of the wage gap in the lower half of the wage distribution, but is insignificant in the upper half.
    \item \cite{folkeSexualHarassmentGender2022}, \cite{batutItManWorld2022}: \textbf{Sexual harassment} pushes women to quit their job for lower paying jobs.
    \item \cite{bamiehCanWageTransparency2022}: women sorting into lower-paying occupations could be due to lack of information about wage in each occupation. Exploit a reform increasing \textbf{transparency} in job ads, but find no effect. \cite{bakerPayTransparencyGender2022}, \cite{bennedsenFirmsRespondGender2022} find the opposite: increase transparency decreases the gap by 30\% / 2p.p. 
    \item \cite{roussilleCentralRoleAsk2022}, \cite{dreberWhyWomenAsk2022}: role of the \textbf{ask gap}: women ask for lower salaries than men when they are hired, which partly drives the gender wage gap. Shows that when candidates are informed on median salary for the position, ask and wage gap vanish. Also \cite{kiesskingGenderDifferencesWage2019}: \textbf{gap in wage expectations}, linked to sorting and negotiation style.
    \item \cite{biasiFlexibleWagesBargaining2022}: \textbf{flexible pay} (where wages are negotiated individually, as opposed to a salary grid) worsens the gender pay gap, possibly due to women negotiating less often.
    \item \cite{cortesGenderDifferencesJob2021}: \textbf{job search} is different between genders: women accept offers earlier than men. The gap narrows over the course of the search period.
    \item \cite{sinWhatDrivesGender2022}: \textbf{Taste discrimination}. \cite{fanfaniTastesDiscriminationMonopsonistic2022}: discimination and monopsony. \cite{birkelundGenderDiscriminationHiring2022} find no discrimination.
    \item \cite{heyneGenderUnemploymentSubjective2022}, \cite{cukrowska-torzewskaGenderGapReservation2021} women's have a higher \textbf{reservation wage} than men's.
  \end{itemize}



  \section{The Motherhood Penalty}

  \begin{itemize}
    \item \cite{klevenChildrenGenderInequality2019}: Children create a 20\% wage gap, driven both by the extensive and intensive margin.
    \item \cite{angelovParenthoodGenderGap2016}: Children drive increase of 10\% in the within couple wage gap.
  \end{itemize}

  \bibliography{biblio}
\end{document}